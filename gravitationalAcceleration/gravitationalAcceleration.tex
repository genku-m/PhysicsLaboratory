\documentclass[a4j,10pt]{jarticle}
\usepackage{graphicx}
\begin{document}
  \section{実験の目的}
  重力加速度を4桁の精度で測定する.
  \section{実験の原理}
  鉛直線を含む一平面内で振動する単振り子の周期は重力加速度に関係しており,周期を$T$,重力加速度を$g$,振り子の長さを$h$とすると,その周期は
    \begin{eqnarray}
      \label{cycle1}
      T = 2 \pi\sqrt{\frac{h}{g}}
    \end{eqnarray}
  となる.この(\ref{cycle1})を使えば振り子の周期$T$を測ることによって重力加速度が求められる.
    \begin{eqnarray}
      \label{gravity}
      g = \frac{4 \pi^2 h}{T^2}
    \end{eqnarray}
  ただしこれらの式は振り子の重りが小さく,振動の振幅も小さい場合の近似値であるので,約$0.1$m/s$^2$の不確かさを持つことになる.
  
  より精密に測るために$T_{0}$ = $2s$の光パルスを用いる.暗い場所で振り子に光パルスを照射し,少し離れたところから望遠鏡で観察すると振り子の吊り線が2秒毎に光って見える.
  周期$T$が正確に$T_{0}$ = $2s$出ないとすると,光って見える吊り線の位置は少しずつずれていく.この吊り線の位置が左右に往復する周期を$\tau$とすると振り子$T$の周期は
    \begin{eqnarray}
      \label{tau1}
      \frac{1}{T_{0}}-\frac{1}{T}=\frac{\pm 1}{\tau}
    \end{eqnarray}
    \begin{eqnarray}
      \label{tau2}
        T = T_{0} \pm \frac{T_{0}^2}{\tau \mp T_{0}}
    \end{eqnarray}
  (\ref{tau1}),(\ref{tau2})より求められる.複号($\pm$ ,$ \mp$)は振り子の周期$T$が$T_{0}$より長ければ上,短ければ下をとる.
  周期の相対的不確かさは$\Delta T/T \cong 5\times10^{-5}$となる.
  
  加えて重りの大きさと振り子の振幅の影響も考慮する.重りの半径$r$,振り子の最大触れ角を$\theta$とすると
    \begin{eqnarray}
    \label{cycle2} 
      T = 2\pi \sqrt{\frac{h}{g} \Bigl( 1+\frac{2 r^2}{5 h^2} \Bigl) }  \, \, \,  \Bigl( 1+\frac{\theta^2}{16} \Bigl)
    \end{eqnarray}
  この式から重力加速度$g$を求めると
    \begin{eqnarray}
      \label{gravity2}
      g = \frac{4 \pi^2 h}{T^2} \Bigl(1 + \frac{2r^2}{5h^2} + \frac{\theta^2}{8} \Bigl)
    \end{eqnarray}
  となる.
  この方法で重力加速度を不確かさ$\pm0.1$cm/s$^2$で求めることができる.
  \section{実験の方法}
  \subsection{振り子の準備}
  まず最初に,球体の直径$d$をノギスで測った.
  次に,金具が周期に影響を与えないように三脚台を水平にしナイフエッジのついた金具をのせ振動の周期が約2秒になっていることを確認した.
  次に,太さ0.2mmのピアノ線で金属球を吊るし,金属球と反対側のピアノ線をナイフエッジのついた金具に固定した.
  次に,ナイフエッジのついた金具を固定台に固定した.
  \subsection{振り子の長さ$h$の測定}
  振り子の長さ$h$の測定するために吊るした金属球と受け皿を接触させ,ナイフエッジから受け皿までの距離$h'$を五回測った.
  振り子の長さ$h$は
    \begin{eqnarray}
      h = h'-\frac{d}{2}
    \end{eqnarray}
  で算出する.
  これを1回目は$h'$が1mよりほんの少し短い状態にし,そこから5mm前後短くしていき,合計3回測定した.
  \subsection{周期$T$と触れ角$\theta$の測定}
  金属球を吊るした状態で金属球からスケールまでの距離$s$を測った.これを利用して
  \begin{eqnarray}
    l = h'-s
  \end{eqnarray}
  を計算した.
  吊るした金属球を振動させた.このときねじれ運動や楕円運動が伴うと精密な測定ができないので最新の注意を払った.
  次に,おおよその周期をストップウォッチを利用し30周期数えて算出した.
  次に,光パルスを照射したピアノ線の位置が基準線にくるのを待ち,そこから光パルスを照射したピアノ線が左右に振れて元の位置に戻るまでの時間を計測した.
  これを3回繰り返した.
  また計測の途中で振幅$a$を計測した.振幅は時間経過とともに小さくなるので,これも3回繰り返して計測した.
  これをそれぞれの長さ分繰り返して測定した.
  測定を終えた最後にもう一度$h'$の長さを測定し,変化していないことを確認した.
  \section{測定結果}
  まず,金属球の直径を測定した$d$とナイフエッジからスケールまでの距離$l$を5回ずつ測定した結果を表1で示す.

  \begin{table}[h]
    \centering
    \caption{$d$,$l$の測定結果}
    \label{dresult}
    \begin{tabular}{c c} \hline
        $d$/mm & $l$/mm \\ \hline
        30.15 & 885.0 \\
        30.15 & 885.0 \\
        30.15 & 885.0 \\
        30.15 & 885.0 \\
        30.15 & 885.0 \\ \hline
    \end{tabular}
  \end{table}
  この結果より平均値を取ると
  \begin{eqnarray*}
    \bar{d} &=& \frac{30.15+30.15+30.15+30.15+30.15+30.15}{5} = 30.15 {\rm mm} \\
    \bar{l} &=& \frac{885.0+885.0+885.0+885.0+885.0+885.0}{5} = 885.0 {\rm mm} \\
  \end{eqnarray*}
  また測定結果からは
  \begin{eqnarray*}
    \Delta d &=& 0 {\rm mm} \\
    \Delta l &=& 0 {\rm mm} \\
  \end{eqnarray*}
  ということになった.

  次に,ナイフエッジから金属球の端までの長さ$h'$の測定結果を、それぞれ$h_1'$,$h_2'$,$h_3'$として表2で示す。
  \begin{table}[h]
    \centering
    \caption{$h'$の測定結果}
    \begin{tabular}{c c c} \hline
        $h_1'$/mm & $h_2'$/mm & $h_3'$/mm \\ \hline
        995.0 & 989.0 & 986.0 \\
        995.0 & 989.0 & 986.0 \\
        995.0 & 989.0 & 986.0 \\
        995.0 & 989.0 & 986.0 \\
        995.0 & 989.0 & 986.0 \\ \hline
    \end{tabular}
  \end{table}
  こちらも平均値を取ると
  \begin{eqnarray*}
    \bar{h_1'} &=& \frac{995.0+995.0+995.0+995.0+995.0+995.0}{5} =  995.0 {\rm mm} \\
    \bar{h_2'} &=& \frac{989.0+989.0+989.0+989.0+989.0+989.0}{5} = 989.0 {\rm mm} \\
    \bar{h_3'} &=& \frac{986.0+986.0+986.0+986.0+986.0+8986.0}{5} = 885.0 {\rm mm} \\
  \end{eqnarray*}
  となった.
  振り子の長さをそれぞれ$h_1$,$h_2$,$h_3$とすると
  \begin{eqnarray*}
    h_1 &=& \bar{h_1'}-\frac{\bar{d}}{2} = 979.92... \cong 979.9\ {\rm mm} \\
    h_2 &=& \bar{h_2'}-\frac{\bar{d}}{2} = 973.92... \cong 973.9\ {\rm mm} \\
    h_3 &=& \bar{h_3'}-\frac{\bar{d}}{2} = 970.92... \cong 970.9\ {\rm mm} \\
\end{eqnarray*}
  それぞれの不確かさは
  \begin{eqnarray*}
  \Delta h_1' &=& 0 {\rm mm} \\
  \Delta h_2' &=& 0 {\rm mm} \\
  \Delta h_3' &=& 0 {\rm mm} \\
  \Delta h_1' &=& 0 {\rm mm} \\
  \Delta h_2' &=& 0 {\rm mm} \\
  \Delta h_3' &=& 0 {\rm mm} \\
  \end{eqnarray*}
  となった.

  次に周期$\tau$と,振り子の振幅$a$を$h_1'$,$h_2'$,$h_3'$にたいしてそれぞれ3回測定したものを表3で示す.
  \begin{table}[h]
    \centering
    \caption{$\tau$,$a$の測定結果}
    \begin{tabular}{c c c} \hline
        & $\tau$/s & $a_{\rm i}$/mm  \\ \hline
        {$h_1$} & 268.68 & 43.0\\
        & 267.73 & 37.0 \\
        & 267.63 & 31.0 \\
        \hline
        {$h_2$} & 207.93 & 43.0\\
        & 207.73 & 35.0 \\
        & 204.25 & 27.0 \\
         \hline
        {$h_3$} & 184.43 & 41.0\\
        & 184.38 & 35.0 \\
        & 179.73 & 31.0 \\
        \hline
    \end{tabular}
\end{table}
$\tau$の値を$h_1$,$h_2$,$h_3$に対応して$\tau_1$,$\tau_2$,$\tau_3$とすると平均値は
\begin{eqnarray*}
  \bar{\tau_1} &=& \frac{268.68+267.73+267.63}{3} =  268.013...\cong 268.01\ {\rm s} \\
  \bar{\tau_2} &=& \frac{207.93+207.73+204.25}{3} = 206.636...\cong 206.64\ {\rm s} \\
  \bar{\tau_3} &=& \frac{184.43+184.38+179.73}{3} = 182.846...\cong 182.85\ {\rm s} \\
\end{eqnarray*}
それぞれの不確かさは
\begin{eqnarray*}
  \Delta \tau_1 &=& \sqrt{ \frac{1}{3\cdot2} \{ (268.68-268.01)^2 + (267.73-268.01)^2 +(267.63-268.01)^2 \} } = 0.345... \cong 0.4\ {\rm s} \\
  \Delta \tau_2 &=& \sqrt{ \frac{1}{3\cdot2} \{ (207.93-206.64)^2 + (207.73-206.64)^2 +(204.25-206.64)^2 \} } = 1.194... \cong 1.2\ {\rm s} \\
  \Delta \tau_3 &=& \sqrt{ \frac{1}{3\cdot2} \{ (184.43-182.85)^2 + (184.38-182.85)^2 +(179.73-182.85)^2 \} } = 2.541... \cong 2.5\ {\rm s} \\
\end{eqnarray*}
となる.
測定結果の$a_{\rm i}$のそれぞれ3回測ったものの最初をそれぞれ$a_i1$,$a_i2$,$a_i3$,最後をそれぞれ$a_f1$,$a_f2$,$a_f3$と置き、$h_1$,$h_2$,$h_3$に対応する$\theta_1^2$,$\theta_2^2$,$\theta_3^2$を計算するとそれぞれ
\begin{eqnarray*}
  \theta_1^2 &=& \frac{a_i1a_f1}{l^2} = 0.00170...\cong 0.0017 \\
  \theta_2^2 &=& \frac{a_i2a_f2}{l^2} = 0.00148...\cong 0.0015 \\
  \theta_3^2 &=& \frac{a_i3a_f3}{l^2} = 0.00162...\cong 0.0016 \\
\end{eqnarray*}

これらの値と,(\ref{tau2})と(\ref{gravity2})の式を利用して,$h_1$,$h_2$,$h_3$における重力加速度を求める.

$h1$の場合,まず(\ref{tau2})の復号の下をとって計算し$T_1$とおくと結果は$T_1=1.98518...\cong1.9852s$となった.
次に$DeltaT_1$を求めると
\begin{eqnarray*}
  \Delta T_1 &=& \frac {T_0^2}{tau_1^2}\Delta tau_1 = 0.00005...\cong 0.0001 \\
\end{eqnarray*}
となる.
次に,(\ref{gravity2})を用いて重力加速度$g_1$を計算すると
\begin{eqnarray*}
  g_1  &=& \frac{4 \pi^2 h_1}{T_1^2} \Bigl(1 + \frac{2r^2}{5h_1^2} + \frac{\theta^2}{8} \Bigl) = 9.8089...\cong9.809\ {\rm m/s}^2 \\
\end{eqnarray*}
となる.
$g_1$の不確かさは
\begin{eqnarray*}
  \Delta g_1= g_1 \sqrt{\Bigl( \frac{\Delta h_1}{h_1} \Bigl)^2+\Bigl( \frac{2\Delta T_1}{T_1}\Bigl)^2 } = 0.0980...\cong0.098\ {\rm m/s}^2  \\
\end{eqnarray*}
となる.

$h2$の場合,まず(\ref{tau2})の復号の下をとって計算し$T_2$とおくと結果は$T_2=1.98046...\cong1.9805s$となった.
次に$DeltaT_2$を求めると
\begin{eqnarray*}
  \Delta T_2 &=& \frac {T_0^2}{tau_2^2}\Delta tau_2 = 0.00009...\cong 0.0001 \\
\end{eqnarray*}
となる.
次に,(\ref{gravity2})を用いて重力加速度$g_2$を計算すると
\begin{eqnarray*}
  g_2  &=& \frac{4 \pi^2 h_2}{T_2^2} \Bigl(1 + \frac{2r^2}{5h_2^2} + \frac{\theta^2}{8} \Bigl) = 9.7950...\cong9.795\ {\rm m/s}^2 \\
\end{eqnarray*}
となる.
$g_2$の不確かさは
\begin{eqnarray*}
  \Delta g_2= g_2 \sqrt{\Bigl( \frac{\Delta h_3}{h_3} \Bigl)^2+\Bigl( \frac{2\Delta T_2}{T_2}\Bigl)^2 } = 0.0975...\cong0.098\ {\rm m/s}^2  \\
\end{eqnarray*}
となる.

$h3$の場合,まず(\ref{tau2})の復号の下をとって計算し$T_3$とおくと結果は$T_3=1.99778...\cong1.998s$となった.
次に$DeltaT_3$を求めると
\begin{eqnarray*}
  \Delta T_2 &=& \frac {T_0^2}{tau_2^2}\Delta tau_2 = 0.00011...\cong 0.0001 \\
\end{eqnarray*}
となる.
次に,(\ref{gravity2})を用いて重力加速度$g_3$を計算すると
\begin{eqnarray*}
  g_3  &=& \frac{4 \pi^2 h_1}{T_1^2} \Bigl(1 + \frac{2r^2}{5h_1^2} + \frac{\theta^2}{8} \Bigl) = 9.5945...\cong9.595\ {\rm m/s}^2 \\
\end{eqnarray*}
となる.
$g_3$の不確かさは
\begin{eqnarray*}
  \Delta g_3= g_3 \sqrt{\Bigl( \frac{\Delta h_3}{h_3} \Bigl)^2+\Bigl( \frac{2\Delta T_3}{T_3}\Bigl)^2 } = 0.0959...\cong0.096\ {\rm m/s}^2  \\
\end{eqnarray*}
となる.

\section{考察}
電気通信大学内で実験を行ったので東京都の重力加速度を参考にする.東京都の重力加速度の値は9.79789${\rm m/s}^2$である.$g_1$の値は不確かさを考慮して9.809$\pm$0.098${\rm m/s}^2$で不確かさの範囲内であり正確に測れている.$g_2$は9.795$\pm$0.098${\rm m/s}^2$となりこれも正確に測れていると言える.
一方で$g_3$の結果は9.595$\pm$0.096${\rm m/s}^2$になり不確かさの範囲内に収まらなく正しく測定できていないという結果になった.
おそらく振り子の揺らし方に問題があり,楕円運動が起こったと考えられる.

また,測定値$h_1'$,$h_2'$,$h_3'$,$d$,$l$がすべて同じ値になっており不確かさの測定ができなくなっている.人間が0.1mmの精度を毎回正確に誤差なく測るのは不可能であるので,揺らぎが出るのが自然であるが不自然な結果になってしまっている.
\section{感想}
考察にもある通り,測定値$h_1'$,$h_2'$,$h_3'$,$d$,$l$がすべて同じというのは今回の物差しを使い目測するという測定方法では自然なものとは言えないだろう.
このような結果になったのは,経験不足もあるが,注意深さが足りなかったと言える.
測定するときには注意深く丁寧に行うべきだと思った.
  

\begin{thebibliography}{9}
\bibitem{kiso} 電気通信大学、『基礎理学実験』2021年、p29-37
\end{thebibliography}
\end{document}