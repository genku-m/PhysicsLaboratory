\documentclass[a4j,10pt]{jarticle}
\usepackage[version=3]{mhchem}
\usepackage[dvipdfmx]{graphicx}
\begin{document}
  \section{10分間テスト}
  1. A. 25 kPa B. 4 kPa C. 8 kPa D. 16kPa E. 5kPa

  2. A. $60^\circ C$  B. $85^\circ C$ C. $83^\circ C$ D. $73^\circ C$ E. $81^\circ C$
  \section{目的}
  理想気体の状態方程式を用いて,気体の分子量を求め状態方程式,蒸気圧曲線の意義を理解する.
  \section{原理}
  圧力を$P$,容積を$V$,その時の絶対温度を$T$としたとき,理想気体の状態方程式は(\ref{PV})のようになる.
  \begin{eqnarray}
    \label{PV}
    PV = nRT
  \end{eqnarray}

  ここで$R$は気体定数,$n$は物質量を指す.分子量$M$の気体の質量が$w$であれば,$n = \frac{w}{M}$であるから,(\ref{PV})は(\ref{M})のようになる.
  \begin{eqnarray}
    \label{M}
    M = \frac{wRT}{PV}
  \end{eqnarray}

  (\ref{M})より,ある質量と圧力のもとで一定容積を占めている気体の質量を測定すると分子量が計算できる.これをデュマ法と呼ぶ.
  なお空気中で秤量を行うときは必ず空気の浮力が生じるため,実際の値よりは見かけ上軽くなる.
  浮力の値は測ろうとするものの体積と同じ体積の空気の質量に等しい.実験では試料気体の蒸気圧と空気の密度から質量の補正を行う.

  \section{実験方法}
  \subsection{準備}
  まず500 mLビーカーに450 mL水を入れ,ブンゼンバーナーで火にかけた.
  十分に乾燥したピクノメーターを栓とともに電子天秤で重量を計測した.
  \subsection{未知試料の測定}
  ピクノメーターにこまごめピペットを用いて未知試料Eを約1.0 mL入れた.
  十分に水が沸騰したビーカーにピクノメーターをくびのあたりまでつけてピクノメーターを固定した.湯が内部に染み込まないように注意した.
  かげろうが観測できなかったため,未知試料が蒸気となる時間を3分と仮定した.
  3分間測ったあとピクノメーターをビーカーから取り出し,水道水で冷却した.このとき水が混入しないように注意した.
  ピクノメーターについた水滴を十分に拭き取り,電子天秤で重量を計測した.
  これをピクノメーターに未知試料をそのまま継ぎ足すかたちで3回繰り返した.
  \subsection{ピクノメーターの容積の測定}
  未知試料の測定で利用したピクノメーターをよく洗い,水をピクノメーターの栓の口の先端まで入れた.
  ピクノメーターの周りの水滴をよく拭き取り,電子天秤で重量を計測した.

  \section{結果}
  十分に乾燥したピクノメーターの重量は34.583 gとなった.
  未知試料の測定結果を表1にまとめる.
  差し引き結果は測定結果から乾燥したピクノメーターの重量を引いたものである.
  \begin{table}[h]
    \begin{center}
      \caption{未知試料の測定結果}
      \begin{tabular}{c|ccccc}
        \hline
        測定回数 & 1 & 2 & 3  \\ \hline
        測定結果 & 34.716 g & 34.826 g & 34.898 g \\
        差し引き結果 & 0.133 g & 0.249 g & 0.315 g \\
        \hline
      \end{tabular}
    \end{center}
  \end{table}
  
  ピクノメーターの容積の測定の測定結果は88.779 gで乾燥したピクノメーターの重量は34.583 gを引くと54.196 gとなる.
  このときの水の温度は19.0度であった.

  実験した日の気圧は1015 hPa,温度は21.5度であった.


  これらの値からまずピクノメーターの容積$V$を求める.\cite{kiso}p76の水の密度より,$d\ce{H2O}$の値は0.099841 $g/cm^{-3}$なので
  \begin{eqnarray}
    \label{V}
    V = \frac{54.196}{0.099841} = 54.2803 cm^{-3}
  \end{eqnarray}
  となった.

  次に乾燥空気の密度$dair$を求める.\cite{kiso}p76の乾燥空気の密度より,セ氏温度21.5度と大気圧761.44036 mmHgを代入すると
  \begin{eqnarray}
    \label{p}
    dair = \frac{0.0012932}{1+0.00367\times21.5}\frac{761.44036/760} = 1.2008\cong 1.201 \times10^{-3} \,g/cm^3
  \end{eqnarray}
  となった.

  これらの値より浮力の補正分$w2$を求める.
  未知試料の蒸気圧を$p$とし\cite{kiso}p77から5kPa,大気圧を$P=1.015\times10^{5}\,Pa$とすると
  \begin{eqnarray}
    \label{w2}
    w2 = \frac{p}{P}\cdot dair \cdot V = \frac{5 \times10^3}{1.015 \times10^5} \cdot 1.201 \times10^{-3} \cdot 54.2803 = 0.00321\cong0.003 g
  \end{eqnarray}
  
  この結果から(\ref{M})の$w$は未知試料の差し引き結果+$w2$となる.
  実験結果の補正後重量を表にすると表 2となる.
  \begin{table}[h]
    \begin{center}
      \caption{未知試料の測定結果}
      \begin{tabular}{c|ccccc}
        \hline
        測定回数 & 1 & 2 & 3  \\ \hline
        重量 & 0.133 g & 0.249 g & 0.315 g \\
        補正後重量 & 0.136 g & 0.252 g & 0.318 g \\
        \hline
      \end{tabular}
    \end{center}
  \end{table}


  (\ref{M})より水温を100度と仮定してそれぞれの分子量を測定した結果をまとめると表 3となった.
  \begin{table}[h]
    \begin{center}
      \caption{未知試料の測定結果}
      \begin{tabular}{c|ccccc}
        \hline
        測定回数 & 1 & 2 & 3  \\ \hline
        未知試料の分子量 & 76.514 & 141.776 & 178.908 \\
        \hline
      \end{tabular}
    \end{center}
  \end{table}

  \section{考察}
  今回の実験で利用した未知試料Eの分子量は46.1であり実験結果は大きくかけ離れたものとなった.
  標準偏差を求めると,24.436となり標準偏差自体もかなり大きな数字となったがどれも$\pm$しても未知試料Eの分子量には近づかなかった.


  なぜこのような結果になったのか行った実験手順から考えると,加熱時間が十分でなかったうえに継ぎ足すかたちで2回目3回目の実験を行ったからだと予想できる.
  今回はかげろうの観測ができなかったので予め時間を3分と決めて加熱したが,未知試料Eがすべて蒸発するには時間が足りなかったのだろう.
  さらにこの状態から継ぎ足すかたちで2回目,3回目と実験を行ったため,蒸発しきれなかった未知試料E分増えていっていることが今回の実験結果から考えられる.

  今後より正確にデュマ法で分子量を測りたいときに気をつけるべきことは,かげろうを観測しやすい状況にすることと,ピクノメーターの中身を毎回できるだけからにすることだと言える.
  また水温も毎回測定するとより良い結果が得られるだろう.

  \begin{thebibliography}{9}
    \bibitem{kiso} 電気通信大学,『基礎科学実験』2021年 p72$\sim$77
    \end{thebibliography}
    \end{document}
  \end{document}
\end{document}