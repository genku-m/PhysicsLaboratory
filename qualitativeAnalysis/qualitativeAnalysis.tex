\documentclass[a4j,10pt]{jarticle}
\usepackage[version=3]{mhchem}
\usepackage[dvipdfmx]{graphicx}
\title{定性分析} 
\author{学籍番号2120029, 氏名 政野玄空}
\date{2021年12月4日}
\begin{document}
\maketitle
  \section{目的}
  7種類の金属イオン(\ce{Ag+},\ce{Pb^2+},\ce{Cu^2+},\ce{Fe^3+},\ce{Al^3+},\ce{Zn^2+},\ce{Ca^2+})のうちいくつかを含む混合溶液を分析し,陽イオンの分離と確認を試みる.

  \section{原理}
  混合溶液から金属イオンを分離したり,検出したりするには,適当な陰イオンを含む溶液を加え一部の金属イオンだけを沈殿させ,沈殿しないものと分離する.

  金属イオンが分属試液で沈殿するかどうかは溶解度積で決まる.
  \section{実験方法}
  \subsection{一般操作}
  \subsubsection{採取}
  未知試料溶液を1$\sim$2 mL程度,試験管に取り,そこから1mLを量り取る.
  \subsubsection{加熱}
  ビーカーに水を入れ,加熱して沸騰させ,その中に試験管を入れる.
  \subsubsection{反応}
  遠心管に試料溶液を入れ,試液を1滴ずつ加え,ガラス棒でよくかき混ぜる.生じた沈殿は一度遠心分離し,液を透明にしてから更に試液を1滴加える.反応がなくなるまで続ける.
  \subsubsection{沈殿の遠心分離}
  沈殿が生じたときは遠心分離器の試料の入った遠心管と反対側に同量の水を入れた遠心管を置いてから3000 rpm以下で1分程度回転させる.
  その上澄液をピペットで吸い上げ,採取する.
  \subsubsection{沈殿の洗浄}
  沈殿の入った遠心管に蒸留水を加え,ガラス棒でよくかき混ぜ遠心分離する.
  \subsubsection{酸性,塩基性の調べ方}
  ガラス棒に液をつけてpH試験紙に染み込ませ,赤に色が変われば酸性,青なら塩基性と判断する.

  \subsection{分離操作}
  未知試料を1 mLとり,6 mol/Lの \ce{HCl}を加え変化を見る.今回は沈殿が生じない.
  
  次に未知試料に1 mol/Lの\ce{CH3CSNH2}を5滴加え,湯浴中で10分加熱し\ce{H2S}が発生させる.その後遠心分離し,発生した沈殿を沈殿1,上澄液を
  上澄液1とする.
  
  ここから沈殿1に対する操作と上澄液1に対する操作を分ける.
  \subsubsection{沈殿1に対する操作}
  沈殿1に6 mol/Lの \ce{HNO3}を1mL加えて蒸発皿に移し煮沸し,さらに3 mol/Lの\ce{H2SO4}を4滴加え煮沸し乾個させる.
  冷やしてから蒸留水を1mL加えて遠心管に移す.このとき沈殿が発生するので遠心分離にかける.このとき発生した沈殿を沈殿2,上澄液を上澄液2とする.
  
  さらに上澄液2に1 mol/Lの\ce{CH2COONH4}を加えて溶かし
  0.5 mol/Lの\ce{K2CrO4}を1滴加える.このとき黄色の沈殿が発生する.
  
  反応式は(\ref{pb})になる.
  \begin{eqnarray}
    \label{pb}
    \ce{Pb^2+ + CrO4^2- -> PbCrO4 v}
  \end{eqnarray}
  
  上澄液2にアルカリ性になるまで6mol/Lの\ce{NH3}を加え続ける.
  このとき溶液は青藍色になる.
  反応式は(\ref{cu})になる.
  \begin{eqnarray}
    \label{cu}
    \ce{Cu(OH)2 + 4NH3 -> [Cu(NH3)4]^2+ + 2OH-}
  \end{eqnarray}
  \subsubsection{上澄液1に対する操作}
  上澄液1に6 mol/Lの\ce{HNO3}を5滴加えて加熱し\ce{H2S}を完全に追い出す.
  \ce{Pb(CH3COO)2)}を1滴染み込ませたろ紙は\ce{H2S}にふれると灰黒色に変化するので,それを利用して確認する.
  淡黄色の沈殿が確認できたら遠心分離して除き,上澄み液にアルカリ性になるまで6 mol/Lの\ce{NH3}を加える.沈殿が生じるので遠心分離する.
  このとき生じた沈殿と上澄液をそれぞれ沈殿3,上澄液3とする.

  沈殿3に対して,6 mol/Lの\ce{NaOH}を5滴加え,さらに1mLの蒸留水を加える.このとき沈殿は発生しない.
  沈殿3に対して,1 mol/Lの\ce{HCl}を徐々に滴下して弱酸性にする.このときゲル状の白色沈殿が確認できる.
  反応式は(\ref{al})になる.

  \begin{eqnarray}
    \label{al}
    \ce{Al^3+ + 3OH-(Na+) -> Al(OH)3 v}
  \end{eqnarray}

  上澄液3に対して\ce{(NH4)2Sx\,}の1滴を加える.湯浴中に数分おき,沈殿の沈むのを待ち,さらに1滴加える.
  沈殿が生じるので遠心分離する.
  このとき生じた沈殿の反応式は(\ref{zn})になる.

  \begin{eqnarray}
    \label{zn}
    \ce{Zn^2+ + S^2- -> ZnS v}
  \end{eqnarray}

  またこのときの上澄液を上澄液4とする.

  上澄液4に6 mol/Lの\ce{HCl}を加えて酸性にし\ce{H2S}を追い出し,酢酸鉛紙を使用し\ce{H2S}がなくなったことを確認する.
  さらに\ce{NH3}を加えアルカリ性にし,1 mol/Lの\ce{(NH4)2CO3}を加える.
  このとき沈殿は発生しない.

  \section{結果}
  今回は実際に実験を行ったわけではないので省略する.
  \section{考察}
  今回は実際に操作を行ったわけではないので,一番はじめに6 mol/Lの \ce{HCl}を加えたときに反応があるかどうかがわからなく化学式を見て検討した結果\ce{AgCl}は発生しないだろうとし,沈殿は発生しないと書いたが,実際はなにかしらの沈殿が発生し,その沈殿物にたいする確認も必要になるのではないかと考えられる.
  また他の箇所も沈殿物が発生していると仮定しているが,発生していない等の可能性もあると考えられる.
  さらに今回の実験では溶液がうまく溶け合わないなどの手順上発生しうるミスなどが全く発生していないという仮定になっている.
  実際には想定通りの色の反応にならないなどもあり得るだろう.

  \begin{thebibliography}{9}
    \bibitem{kiso} 電気通信大学,『基礎科学実験』2021年 p40$\sim$49
    \end{thebibliography}
    \end{document}
  \end{document}
\end{document}